\documentclass{article}

\usepackage{polski}
\usepackage[utf8]{inputenc}
\usepackage{graphicx}
\graphicspath{ {./images/} }
\usepackage{amsfonts}
\usepackage{amssymb}
\usepackage{amsmath}
\usepackage{listings}
\usepackage{breqn}
\usepackage{float}

\author{Piotr Koproń \and Mikołaj Leonhardt \and Jan Augustyn}
\date{Maj 2023}

\title{Algorytmy Inspirowane Biologicznie - algorytm genetyczny "Snake"}
\begin{document}
\maketitle
\newpage
\section{Internal Notes - remove when done}
Jak odpalać żeby działał itp itd.
\section{Abstrakt}
Teoria "co robimy"
\section{Implementacja}
Jak zaimplementowaliśmy tego sneka.
\section{Rezultaty nauki}
Czego się nauczył?
\paragraph{Parametry konfiguracyjne}
Co możemy zmieniać (np. wielkość mapy, długość nauki, itp itd.)
\subsection{Rezutaty z zestawu parametrów 1}
\paragraph{Zestaw parametrów 1}
\paragraph{Rezutaty nauki}
Ładne grafy i komentarze tutaj.
\subsection{Rezutaty z zestawu parametrów 1}
\paragraph{Zestaw parametrów 1}
\paragraph{Rezutaty nauki}
Ładne grafy i komentarze tutaj.
\section{Wnioski} 
Jeżeli na jakieś wpadniemy.
\end{document}
